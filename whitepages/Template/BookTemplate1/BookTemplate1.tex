\documentclass[10pt,letterpaper,openany,bibliography=totoc]{book}
\usepackage[utf8]{inputenc}
\usepackage{amsmath}
\usepackage{amsfonts}
\usepackage{amssymb}
\usepackage{graphicx}
\usepackage{cancel}
\usepackage{float}
\usepackage{cite}

\usepackage[ruled,vlined]{algorithm2e}


\usepackage[left=0.75in, 
                    right=0.75in, 
                    bottom=1.0in,
                    top=0.75in]{geometry}


%\usepackage{caption} 
%\captionsetup[table]{skip=10pt}
%\usepackage[font=small,labelfont=bf]{caption}

\usepackage{comment}
\usepackage{listings}

\usepackage{color}
\definecolor{Brown}{cmyk}{0,0.81,1,0.60}
\definecolor{OliveGreen}{cmyk}{0.64,0,0.95,0.40}
\definecolor{CadetBlue}{cmyk}{0.62,0.57,0.23,0}

\usepackage{multicol}

\usepackage{appendix}

\usepackage{fancyhdr}
\usepackage[colorlinks=false,
                     linkcolor=blue,
                     urlcolor=black,
                     bookmarksopen=true,
                     bookmarks,
                     hidelinks]{hyperref}
\usepackage{bookmark}
\usepackage{minitoc}

%================== List of figures and tables mods
\usepackage{tocloft}
\usepackage[labelfont=bf]{caption}

\renewcommand{\cftfigpresnum}{Figure\ }
\renewcommand{\cfttabpresnum}{Table\ }

\newlength{\mylenf}
\settowidth{\mylenf}{\cftfigpresnum}
\setlength{\cftfignumwidth}{\dimexpr\mylenf+3.5em}
\setlength{\cfttabnumwidth}{\dimexpr\mylenf+1.5em}

\usepackage{lipsum}
\usepackage{titlesec}
\titleformat{\chapter}[hang]
{\normalfont\huge\bfseries}{\chaptertitlename\ \thechapter}{20pt}{\Huge}   
\titlespacing*{\chapter}{0pt}{-0.25in}{0.125in}

%============================= Put document title here
\newcommand{\DOCTITLE}{Document Title}  

%=============================  Load list of user-defined commands
% Mark URL's
\newcommand{\URL}[1]{{\textcolor{blue}{#1}}}
%
% Ways of grouping things
%
\newcommand{\bracket}[1]{\left[ #1 \right]}
\newcommand{\bracet}[1]{\left\{ #1 \right\}}
\newcommand{\fn}[1]{\left( #1 \right)}
\newcommand{\ave}[1]{\left\langle #1 \right\rangle}
\newcommand{\norm}[1]{\Arrowvert #1 \Arrowvert}
\newcommand{\abs}[1]{\arrowvert #1 \arrowvert}
%
% Derivative forms
%
\newcommand{\dxdy}[2]{\frac{\partial #1}{\partial #2}}
\newcommand{\dxy}[2]{\frac{d #1}{d #2}}
\newcommand{\dydx}[1]{\frac{\partial #1}{\partial x}}
\newcommand{\dydt}[1]{\frac{\partial #1}{\partial t}}
\newcommand{\dxdz}[1]{\frac{\partial #1}{\partial z}}
\newcommand{\dfdt}[1]{\frac{\partial}{\partial t} \fn{#1}}
\newcommand{\dfdz}[1]{\frac{\partial}{\partial z} \fn{#1}}
\newcommand{\ddt}[1]{\frac{\partial}{\partial t} #1}
\newcommand{\ddz}[1]{\frac{\partial}{\partial z} #1}
\newcommand{\dd}[2]{\frac{\partial}{\partial #1} #2}
\newcommand{\ddx}[1]{\frac{\partial}{\partial x} #1}
\newcommand{\ddy}[1]{\frac{\partial}{\partial y} #1}
\newcommand{\dxdyn}[3]{\frac{\partial ^{#3} #1 }{\partial #2 ^{#3}}}
\newcommand{\Dxdy}[2]{\frac{D #1}{D #2}}
\newcommand{\Dxy}[2]{\frac{D #1}{D #2}}
%
% Bold quantities
% 
\newcommand{\Omegabf}{\mathbf{\Omega}}
\newcommand{\bnabla}{\boldsymbol{\nabla}}
\newcommand{\position}{\mathbf{x}}
\newcommand{\dotp}{\boldsymbol{\cdot}}
%
% Vector forms
%
\renewcommand{\vec}[1]{\mbox{$\stackrel{\longrightarrow}{#1}$}}
\renewcommand{\div}{\mbox{$\vec{\mathbf{\nabla}} \cdot$}}
\newcommand{\grad}{\mbox{$\vec{\mathbf{\nabla}}$}}
\newcommand{\bb}[1]{\bar{\bar{#1}}}
%
% Vector forms boldfaced
\newcommand{\bvec}[1]{\mathbf{#1}}
\newcommand{\bdiv}{\boldsymbol{\nabla} \boldsymbol{\cdot}}
\newcommand{\bgrad}{\bnabla}
\newcommand{\mat}[1]{\bar{\bar{#1}}}
%
%
% Equation beginnings and endings
%
% Un-numbered equation with alignment
\newcommand{\beq}{\begin{equation*} \begin{aligned}}
\newcommand{\eeq}{\end{aligned}\end{equation*}}
% Numbered equation with alignment
\newcommand{\beqn}{\begin{equation}\begin{aligned}}
\newcommand{\eeqn}{\end{aligned}\end{equation}}  

% Numbered equation array, aligned	
\def\bea#1\eea{\begin{align}#1\end{align}}

\newcommand{\beas}{\begin{eqnarray*}}
\newcommand{\eeas}{\end{eqnarray*}}
\newcommand{\bdm}{\begin{displaymath}}
\newcommand{\edm}{\end{displaymath}}
%
% Equation punctuation
%
\newcommand{\pec}{\, ,}
\newcommand{\pep}{\, .} 
\newcommand{\pev}{\hspace{0.25in}}
%
% Equation labels and references, figure references, table references
%
\newcommand{\lequ}[1]{\label{eq:#1}}
\newcommand{\equ}[1]{Eq.~(\ref{eq:#1})}
\newcommand{\equs}[1]{Eqs.~(\ref{eq:#1})}
\newcommand{\requ}[1]{(\ref{eq:#1})}
\newcommand{\lfig}[1]{\label{fi:#1}}
\newcommand{\fig}[1]{Fig.~\ref{fi:#1}}
\newcommand{\figs}[1]{Figs.~\ref{fi:#1}}
\newcommand{\rfig}[1]{\ref{fi:#1}}
\newcommand{\lta}[1]{\label{ta:#1}}
\newcommand{\ta}[1]{Table~\ref{ta:#1}}
\newcommand{\rta}[1]{\ref{ta:#1}}
\newcommand{\lsec}[1]{\label{sec:#1}}
\newcommand{\rsec}[1]{\ref{sec:#1}}
%
% Superscript and subscript in text
%
\newcommand{\supertext}[1]{\ensuremath{^{\textrm{#1}}}}
\newcommand{\subtext}[1]{\ensuremath{_{\textrm{#1}}}}
%
% List beginnings and endings
%
\newcommand{\bl}{\bss\begin{itemize}}
\newcommand{\el}{\vspace{-.5\baselineskip}\end{itemize}\ess}
\newcommand{\ben}{\bss\begin{enumerate}}
\newcommand{\een}{\vspace{-.5\baselineskip}\end{enumerate}\ess}
%
% Figure and table beginnings and endings
%
\newcommand{\bfg}{\begin{figure}}
\newcommand{\efg}{\end{figure}}
\newcommand{\bt}{\begin{table}}
\newcommand{\et}{\end{table}}
%
% Tabular and center beginnings and endings
%
\newcommand{\bc}{\begin{center}}
\newcommand{\ec}{\end{center}}
\newcommand{\btb}{\begin{center}\begin{tabular}}
\newcommand{\etb}{\end{tabular}\end{center}}
%
% Single space command
%
\newcommand{\bss}{\begin{singlespace}}
\newcommand{\ess}{\end{singlespace}}
%
% Quick commands for symbols
%
\newcommand{\half}{\frac{1}{2}}
\newcommand{\third}{\frac{1}{3}}
\newcommand{\twothird}{\frac{2}{3}}
\newcommand{\fourth}{\frac{1}{4}}
\newcommand{\sixth}{\frac{1}{6}}
\newcommand{\mdot}{\dot{m}}
%\newcommand{\ten}[1]{\times 10^{#1}\,}
\newcommand{\cL}{{\cal L}}
\newcommand{\cD}{{\cal D}}
\newcommand{\cF}{{\cal F}}
\newcommand{\cE}{{\cal E}}
\renewcommand{\Re}{\mbox{Re}}
\newcommand{\Ma}{\mbox{Ma}}
\newcommand{\mA}{\mathbf{A}}
\newcommand{\mB}{\mathbf{B}}
\newcommand{\mC}{\mathbf{C}}
\newcommand{\E}{\mathcal{E}}
\newcommand{\F}{\mathcal{F}}
\newcommand{\Q}{\mathcal{Q}}
\newcommand{\U}{\mathbf{U}}
\renewcommand{\H}{\mathbf{H}}
\newcommand{\R}{\mathbf{R}}
\newcommand{\SN}{S$_{n}\,$}
\newcommand{\Flux}{\mathbf{F}}
\newcommand{\dt}{\Delta t}
\newcommand{\dx}{\Delta x}
\newcommand{\iL}{_{i,L}}
\newcommand{\iR}{_{i,R}}
\newcommand{\sa}{\sigma_a}
\newcommand{\sigsL}{\frac{\sigma_{s,i,L}^k}{2}}
\newcommand{\sigsR}{\frac{\sigma_{s,i,R}^k}{2}}
\newcommand{\sigtL}{\sigma_{t,i,L}^k}
\newcommand{\sigtR}{\sigma_{t,i,R}^k}
\newcommand{\halfh}{\frac{h_i}{2}}
\newcommand{\CN}[3]{\half\left[#1\right]^#2 + \half\left[#1\right]^#3}
\newcommand{\CNN}[3]{\half\left[#1\right]^#2 - \half\left[#1\right]^#3}
\newcommand{\BDF}[4]{\sixth\left[#1\right]^{#2} + \sixth\left[#1\right]^{#3} + \twothird\left[#1\right]^{#4}}
%
% More Quick Commands
%
\newcommand{\bi}{\begin{itemize}}
\newcommand{\ei}{\end{itemize}}
\newcommand{\dxi}{\Delta x_i}
\newcommand{\dyj}{\Delta y_j}
\newcommand{\ts}[1]{\textstyle #1}

\newcommand{\jcr}[1]{\textcolor{magenta}{#1}}
\usepackage[normalem]{ulem}
\newcommand{\ssout}[1]{\sout{\textcolor{magenta}{#1}}}

%
% Code syntax highlighting
%
%\lstset{language=C++,frame=ltrb,framesep=2pt,basicstyle=\linespread{0.8} \small,
%	keywordstyle=\ttfamily\color{OliveGreen},
%	identifierstyle=\ttfamily\color{CadetBlue}\bfseries,
%	commentstyle=\color{Brown},
%	stringstyle=\ttfamily,
%	showstringspaces=true,
%	tabsize=2,}

\lstset{language=C++,frame=ltrb,framesep=8pt,basicstyle=\linespread{0.8} \Large,
commentstyle=\ttfamily\color{OliveGreen},
keywordstyle=\ttfamily\color{blue},
identifierstyle=\ttfamily\color{CadetBlue}\bfseries,
stringstyle=\ttfamily,
tabsize=2,
showstringspaces=false,
numbers=left,
captionpos=t}

\renewcommand{\lstlistingname}{\textbf{Code Snippet}}% Listing -> Code Snippet


\begin{document}
\frontmatter
\begin{titlepage}
	\pagestyle{fancy}
	\vspace*{1.0cm}
	\centering
	\vspace{1cm}
	\vspace{.25cm}
	{\Large\bfseries \DOCTITLE  \par}
	\vspace{1cm}
	{\Large August, 2019 \par}
	\vspace{1.0cm}
	{\Large Jan Vermaak \par}
	{\Large 1st Edition \par}
	\begin{center}
			\begin{minipage}[c]{0.45\textwidth}
				\begin{figure}[H]
					
					\includegraphics[width=3in]{Logo2_Medium.png}
				\end{figure}
			\end{minipage}
		\end{center}

\end{titlepage}	

\newpage 
\setcounter{page}{1}

\noindent
{\LARGE\textbf{\DOCTITLE}}
\newline
\newline
\newline
\noindent
{\Large John K. Cena${^1}$, Dwayne T.R. Johnson$^{1,2}$}
\newline
\noindent\rule{\textwidth}{1pt}
{\small $^1$Center for Large Scale Scientific Simulations, Texas A\&M Engineering Experiment Station, College Station, Texas, USA.}
\newline\noindent
{\small $^2$Nuclear Engineering Department, Texas A\&M University, College Station, Texas, USA.}
\newline
\newline
\textbf{Abstract:}\newline\noindent
Work is work for some, but for some it is play.
\newline
\newline\noindent
{\small
\textbf{Keywords:} transport sweeps; discrete-ordinate method; radiation transport; massively parallel simulations; discontinuous Galerkin; unstructured mesh}

\newpage
\dominitoc
\tableofcontents
\addtocontents{toc}{~\hfill\textbf{Page}\par}

\newpage
\mainmatter
\setcounter{chapter}{0}
\chapter{Introduction}
\minitoc
\section{Introduction}
For equations you can use the following shortcuts:

\begin{verbatim}
% Un-numbered equation with alignment
\beq 
x^2 + y^2 &= R^2 \\
y         &= mx + c 
\eeq 
\end{verbatim}
\beq 
x^2 + y^2 &= R^2 \\
y &= mx + c 
\eeq 
\begin{verbatim}
% Numbered equation with alignment
\beqn 
x^2 + y^2 &= R^2 \\
y         &= mx + c 
\eeqn 
\end{verbatim}
\beqn 
x^2 + y^2 &= R^2 \\
y &= mx + c 
\eeqn

 \begin{verbatim}
 % Numbered equation array with alignment
\begin{align}
x^2 + y^2 &= R^2 \\
y         &= mx + c 
 \end{align}
 \end{verbatim}

\bea
 x^2 + y^2 &= R^2 \\
 y &= mx + c 
\eea
Vector notations:

\begin{table}
\centering
\begin{tabular}{ p{1.5in}  p{1.5in}}
\beq 
\text{\textbackslash div} &= \div \\
\text{\textbackslash grad} &= \grad \\
\text{\textbackslash vec\{x\}} &= \vec{x}\\
\text{\textbackslash bb}\{A\} &= \bb{A}
\eeq 
&
\beq 
\text{\textbackslash bdiv} &= \bdiv \\
\text{\textbackslash bgrad} &= \bgrad \\
\text{\textbackslash bvec\{x\}} &= \bvec{x}\\
\text{\textbackslash mat\{A\}} &= \mat{A} \\
\text{\textbackslash Omegabf} &= \Omegabf
\eeq 
\end{tabular}
\end{table}

\begin{figure}[H]
\centering
\includegraphics[width=0.3\linewidth]{Logo2_Medium.png}
\caption{Example figure inclusion}
\label{fig:logo2medium}
\end{figure}


Code highlighting C++:
\begin{lstlisting}[language=c++,caption={Code example}]
// This is a single line comment
/*This is a
multiline comment.*/
int main(int argc, char** argv)
{
	double x=2;
	std::cout << "hello world";
	FunctionCall(2);
	return 0;
}
\end{lstlisting}
\begin{lstlisting}[language=c++,caption={Code example smaller},
                            basicstyle=\linespread{0.8} \small]
// This is a single line comment
/*This is a
multiline comment.*/
int main(int argc, char** argv)
{
	double x=2;
	std::cout << "hello world";
	FunctionCall(2);
	return 0;
}
\end{lstlisting}


\section{Conclusions and Outlook}
This work is obviously the most awesome but maybe someone might want to look at closing the valve on section 4. It might open a portal to another dimension.
asd
\newpage
\section{Acknowledgments}
The computing resources available in this work was made possible by involvement in Texas A\&M's Center for Exascale Radiation Transport which was supported by the Department of Energy, National Nuclear
Security Administration, under Award Number(s) DE-NA0002376. Established by Congress in 2000, NNSA is a semi-autonomous agency within the U.S. Department of Energy responsible for enhancing national security through the military application of nuclear science. NNSA maintains and enhances the safety, security, reliability and performance of the U.S. nuclear weapons stockpile without nuclear testing; works to reduce global danger from weapons of mass destruction; provides the U.S. Navy with safe and effective nuclear propulsion; and responds to nuclear and radiological emergencies in the U.S. and abroad.
One of the authors (JR) was partially funded through a grant by the Department of the Defense, Defense Threat Reduction Agency under Award No. HDTRA1-18-1-0020. The content of the information does not necessarily reflect the position or the policy of the federal government, and no official endorsement should be inferred.

\chapter{Cool stuff A}
\minitoc
\section{Orbital mechanics}


\chapter{More}
\minitoc
\section{What more?}


\newpage
\begin{thebibliography}{1}
	
	\bibitem{LewisMiller} Lewis E.E., Miller W.F., {\em Computational Methods of Neutron Transport}, JohnWiley \& Sons, 1984
	   
\end{thebibliography}

\newpage
\begin{appendices}
\chapter{First appendix}
Put ``Lazy reader stuff here".
\end{appendices}

\end{document}